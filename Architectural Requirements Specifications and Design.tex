\documentclass{article}

\usepackage[margin=2.5cm,left=2cm,includefoot]{geometry}
\usepackage{hyperref}
\usepackage{array}
\usepackage{enumitem}
\usepackage{graphicx}
\usepackage[section]{placeins}
\usepackage{titlesec}
\usepackage{float}

% Set depth for sections (one more than usual)
\setcounter{secnumdepth}{4}

% Set paragraphs to be the 4th section depth
\titleformat{\paragraph}
{\normalfont\normalsize\bfseries}{\theparagraph}{1em}{}
\titlespacing*{\paragraph}
{0pt}{3.25ex plus 1ex minus .2ex}{1.5ex plus .2ex}

% Header and footer
\usepackage{fancyhdr}
\pagestyle{fancy}

\rhead{COS301 - \LaTeX}
\lhead{Team Algol}
\fancyfoot{}
\fancyfoot[R]{Page \thepage}

\renewcommand{\headrulewidth}{2pt}
\renewcommand{\footrulewidth}{1pt}
%

\begin{document}

	\begin{titlepage}
		\begin{center}

			\begin{figure}
				\includegraphics[width=\linewidth]{images/logo.jpg}	
			\end{figure}

			\line(1,0){400}\\
			[6mm]
			\huge{
				\bfseries Architectural Requirements Specifications and Design
			}\\
			[2mm]
			\line(1,0){300}\\
			[10mm]
			\textsc{\large NavUP}\\
			[7.5mm]
			\textsc{\large University of Pretoria - Team Algol}\\
			[10mm]
			\large{\textbf{Created By:}}\\
			[2mm]
			% Add your own details here 
			\href{https://github.com/KeatonPennels}{Keaton Pennels - 14373018}\\
			\href{https://github.com/Chris19951225}{Hristian Vitrychenko - 15006442}\\
      \href{https://github.com/lindelo}{Lindelo Mapumulo - 12002862}\\
			\href{https://github.com/DuartBreedt}{Duart Breedt - 15054692}\\
			\href{https://github.com/roberttrankle}{Robert Trankle - 15092454}\\[5mm]


		\href{https://github.com/Chris19951225/COS-301-Team-Algol}{\textsc{\Large GitHub Repository - Team Algol}}
		\\
		\href{https://github.com/Chris19951225/COS-301-Team-Algol}{\textsc {For more information, please click here}}
		  
		\end{center}
		\begin{flushright}
			\textsc{\large 26 February 2017}
		\end{flushright}
	\end{titlepage}

	\cleardoublepage
	\thispagestyle{empty}
	\tableofcontents
	\cleardoublepage

	\thispagestyle{empty}
	\listoffigures
	\cleardoublepage
	\setcounter{page}{1}
	
	
	\section{Introduction}\label{sec:intro}
		This chapter of the document aims to identify the architectural design specifications which satisfy the identified functional requirements of the NavUP system. Additionally, it will include any terms, abbreviations, acronyms and references used throughout this document.
	
		\subsection{Purpose}\label{subsec:purpose}
			The main purpose of this document is firstly identifying the subsystem architectural design requirements, constraints, integration need  and then addressing these elemts by using architectural patterns and diagrams to model system behaviour and interactions. The choice of appropriate technologies for the implementation of this system that satisfy the various requirements will also be prosposed along with motivations for these propositions 
				
		\subsection{Definitions, Acronyms, and Abbreviations}\label{subsec:daa}
			\begin{table}[h!]
				\centering
				\caption{Table of Definitions, Acronyms, and Abbreviations used in this document}
				\label{tab: Table 1}
				\begin{tabular}{| m{4cm} | m{12cm} |}
					\hline
					\textbf{Term} & \textbf{Definition} \\
					\hline
					\hline
					
				\end{tabular}
			\end{table}
	
		\subsection{References}\label{subsec:references}
		List each document referenced in the SRS by title, report number, date, publisher, and where and how to get it
		
	\section{Requirements and Constraints}\label{sec:requirements}

	The following section provides all the requirements of NavUP, and possible constraints regarding the mobile application.
			\subsection{External Interface Requirements}\label{subsec:external}
			This section provides a detailed description of all the inputs and outputs of the system. Descriptions of the user, hardware, software and communication interfaces are also provided.
			\subsubsection{User interfaces}
			Upon opening the application, a first time user (or a user that had logged out) should see the splash page - Figure 1. The splash page will allow a user to continue as a guest (without logging in) or login in as a user. The user can either swipe left to login or right to use the application as a guest. \\
			
			The log in page (Figure 2) should prompt the user to enter their login details (user name \& password). If the user has not yet created an account, they will be prompted to register a new account before they can use or continue with the application. The user register page bears resemblance to the login page prototype, but includes input fields for the user's email address and two password fields (one for the actual password, and another to verify the previously entered password). \\
			
			In addition to being able to register as a new user, a user should be able to view, update and/or delete their profile. The prototype of this page is shown on Figure 3; on this page, a user can perform CRUD operations (excluding CREATE) on their profiles. Admin profiles will have additional options (e.g. making a normal user an administrator user). These options will appear in the "User Details" page.
\begin{figure}[H]
    \minipage{0.32\textwidth}
	\centering{\includegraphics[width=0.5\linewidth]{./images/Splash_Screen.JPG}}
	\caption{Splash Screen}
	\endminipage\hfill
	\minipage{0.32\textwidth}
	\centering{\includegraphics[width=0.5\linewidth]{./images/Login_Screen.JPG}}
	\caption{Login Screen}
	\endminipage\hfill
	\minipage{0.32\textwidth}
	\centering{\includegraphics[width=0.5\linewidth]{./images/User_Details.JPG}}
	\caption{User Details Screen}
	\endminipage\hfill
\end{figure}

A user should be able to find the directions between two (or more) points. The user should also be provided with additional features to improve the navigation between these points, these include but are not limited to the shortest route. Figure 4 shows the search page to find directions between two points, and Figure 5 shows the results and further filtering options to improve the nature of the directions.\\

\begin{figure}[H]
    \minipage{0.3\textwidth}
	\centering{\includegraphics[width=0.5\linewidth]{./images/Get_Directions_Search.JPG}}
	\caption{Get Directions Search Screen}
	\endminipage\hfill
	\minipage{0.3\textwidth}
	\centering{\includegraphics[width=0.5\linewidth]{./images/Get_Directions_Results.JPG}}
	\caption{Get Directions Result Screen}
	\endminipage\hfill
	\minipage{0.3\textwidth}
	\centering{\includegraphics[width=0.5\linewidth]{./images/POI_Results.JPG}}
	\caption{Search Locations Result(s) Screen}
	\endminipage\hfill
\end{figure}

Users should also be able to get their current location, save various points of interest and search for a location. Figure 6 shows the prototype page for the results produced after a user has searched for a location. \\

A user should also be able to send a notification. The notification page should have a similar layout as the other pages shown across Figure 1 - 6.
			\subsubsection{Hardware interfaces}
			Since NavUP uses existing hardware (i.e. smart devices), it does not require direct hardware interfaces other than the interfaces that are provided by the user's device. The GPS features are explicitly managed by the user's device, particularly the operating system on which the device runs.
			\subsubsection{Software interfaces}
			NavUP communicates with the device's GPS to get data about the user's location and the other geographical information regarding the user's location. Furthermore, the application communicates with the on-campus wireless hotspots to tag the relative location of a user within a building. NavUP will mostly just perform read operations, which are communicated to the application via the device's operating system.
			\subsubsection{Communication interfaces}
			The application will be heavily dependent on the device's GPS; without this communication NavUP will be unable to perform any location specific operations (including getting the user's current location). The application also needs a working wireless network connection in order to facilitate communication with the GPS's remote database. The wireless hotspots within the university will also be required to get the user's relative building location; without this, it may be difficult to provide accurate results.

			\subsection{Performance Requirements}\label{subsec:performance}
				This section provides a detailed specification of the user's interaction with NavUP, as well as the measures put in place to ensure good system performance.
			
			\subsubsection{Intuitive user interface}
		    \begin{itemize}
		    \item[]TITLE      : Intuitive user interface
		    \item[]DESCRIPTION: The user interface for NavUP should be intuitive and not at any stage confuse users.
		    \item[]RATIONAL   : This ensures easy navigation for the user when using NavUP.
		    \item[]DEPENDENCY : No dependencies.
		    \end{itemize}
			\subsubsection{Intuitive search feature}
			\begin{itemize}
		    \item[]TITLE      : Intuitive search feature
		    \item[]DESCRIPTION: The user should be provided with clear search results when trying to find locations.
		    \item[]RATIONAL   : This ensures that users are not confused by the search results, and can understand the details returned by the search results.
		    \item[]DEPENDENCY : No dependencies.
		    \end{itemize}	
			\subsubsection{Intuitive map view}
			\begin{itemize}
		    \item[]TITLE      : Intuitive map view
		    \item[]DESCRIPTION: The map view presented to the user should be easy to understand and leave no room for ambiguous interpretations.
		    \item[]RATIONAL   : This ensures that the user will be able to use the map view to navigate effectively.
		    \item[]DEPENDENCY : No dependencies.
		    \end{itemize}	
			\subsubsection{Intuitive information/help feature}
			\begin{itemize}
		    \item[]TITLE      : Intuitive information/help feature
		    \item[]DESCRIPTION: The information buttons (seen in the search results) - and help features - should be easy to find and use.
		    \item[]RATIONAL   : This ensures that users can use information and help features easily, since the users are most likely to interact with these features when using NavUP for the first time.
		    \item[]DEPENDENCY : No dependencies.
		    \end{itemize}	
			\subsubsection{Fast response time}
			\begin{itemize}
		    \item[]TITLE      : Fast response time
		    \item[]DESCRIPTION: Results for getting routes or finding locations should not take much time.
		    \item[]SCALE      : The time it takes to get a route and find a location.
		    \item[]METER      : Measured by finding 100 routes and locations, during the testing phase.
		    \item[]MUST       : Must take no more than 3 seconds 100\% of the time.
		    \item[]WISH       : Preferably shouldn't take longer than a second, 100\% of the time.  
		    \end{itemize}
			\subsubsection{Dependable system}
			\begin{itemize}
		    \item[]TITLE      : Dependable system
		    \item[]DESCRIPTION: NavUP should be able to deal with (recover from) all faults.
		    \item[]SCALE      : Should a user incorrectly enter search criteria or NavUP not be able to establish an internet connection, a notification should be sent to a user.
		    \item[]METER      : Measured by using NavUP for at least 10 hours, during the testing phase.
		    \item[]MUST       : NavUP must be dependable at all times.
		    \end{itemize}	
						
			\subsection{Design Constraints}\label{subsec:design constraints}	
				This section provides the design constraints on NavUP's software, which is caused by the device's hardware.
				
		    \subsubsection{Hard disk drive space}
			\begin{itemize}
		    \item[]TITLE      : Hard disk drive space
		    \item[]DESCRIPTION: The amount of space NavUP occupies on installation.
		    \item[]SCALE      : Megabyte(MB).
		    \item[]METER      : Monitoring the space occupation during the testing phase.
		    \item[]MUST       : NavUP shouldn't occupy more than 100MB.
		    \item[]WISH       : Preferably, NavUP shouldn't occupy more than 50MB.
		    \end{itemize}
		    
		    \subsubsection{Application RAM (Random-Access-Memory) usage}
			\begin{itemize}
		    \item[]TITLE      : Application RAM (Random-Access-Memory) usage
		    \item[]DESCRIPTION: The amount of RAM NavUP uses while running on the device. 
		    \item[]SCALE      : Megabyte (MB).
		    \item[]METER      : Monitoring RAM usage during the testing phase.
		    \item[]MUST       : NavUP shouldn't use more than 50MB of the device's RAM.
		    \item[]WISH       : Preferably, NavUP shouldn't use more than 30MB of the device's RAM.
		    \end{itemize}
			\subsection{Software System Attributes}\label{subsec:attributes}

		\newpage

		\section{Module Design}\label{sec:moduels}	
      \subsection{User Management Subsystem}\label{subsec:users}
      	\begin{figure}[H]
			\includegraphics[scale=0.5]{Diagrams/User_Class_Diagram.png}
			\caption{User Management Model Class Diagram}
			\label{fig:User_class}
	\end{figure}
      
      		\begin{figure}[H]
                    \centering \includegraphics[height=0.85\textheight]{UMM-Diagrams/UMM-ActivityDiagram}
                     \caption{User Management Model Activity Diagram}
					 \label{fig:user_activity}
			    \end{figure}
			    {The activity diagram for the User Management Model is displayed in the figure above. It illustrates the registration and login steps of the system, when a user connects as a guest.egistration and login steps of the system, when a user connects as a guest. \\\\
				Initially the guest user will connect to the service/system, it will then be able to login or register. If the User chooses neither of these option it will be displayed the Public services of the system, and it will not be able to access the attributes of being a registered user.\\\\
				If the user wishes to register/login then it will be displayed the register/login page. A registered user will need to enter its credentials to login and if the correct information is entered then it will display the logged-in page, where it can view its information. If the incorrect information is entered, an error message will be displayed and the user will be able to re-enter its credentials.\\\\
				If a user wishes to register, the system will display the registration page and the user will need to enter in the registration information.  The system will check if the email is registred and if it is then it will display an error message. If the email is not registered it will create the account and notify the user that its account has been created.\\\\
				If the user is an admin, it will be able to access the admin rights from the logged-in page.}
				\begin{figure}[H]
                    \centering \includegraphics[height=0.55\textheight]{UMM-Diagrams/UMM-SequenceDiagram}
                     \caption{User Management Model Sequence Diagram}
					 \label{fig:user_sequence}
			    \end{figure}
			    {This Diagram shows the sequence of events the system will use to register a user. The Controller class will first authenticate the email address, and check that it is not currently in the system already and that it is a valid email address. Once this is done the Controller class will use the Participant class to register the user with a name, surname, email address and password. The Participant class will then create the user and set the appropriate fields in the User class. After this is done a boolean variable will be sent back to the controller class indicating if the registration was successful.\\\\
			    The System will then encrypt and store the password on its database, this is to ensure that the user password cannot be compromised.}
			    \begin{figure}[H]
                    \centering \includegraphics[height=0.55\textheight]{UMM-Diagrams/UMMUseCase}
                     \caption{User Management Model Use Case Diagram}
					 \label{fig:user_usecase}
			    \end{figure}
			    {The user and admin will be inherited from the guest actor. If a guest chooses to register it will be notified by the system, this extends the registration process. For the CRUD information of a user, a user will be able to display a route as well as view the distance it walked.}
			    
			\subsection{Navigation Subsystem}\label{subsec:navigation}
			
				\begin{figure}[H]
					\includegraphics[scale=0.5]{Diagrams/Class_Diagram_Navigation.JPG}
					\caption{Navigation Subsystem Class Diagram}	
				\end{figure}
				{Using the Composite design partern by providing the interface to which the user interacts with, the navigation class diagram shows how the user object interacts with other navigation  modules i.e the location module used to returns the users current location using the getCurrent location function and the route access that get and or calculates the route using waypoints returned from the waypoints class.\\\\}
			
				\begin{figure}[H]
					\includegraphics[scale=0.3]{Diagrams/Activity_Diagram_Navigation.JPG}
					\caption{Navigation Activity Diagram}	
				\end{figure}
				{This Diagram models the activity of the interaction for navifation during the users whole navigation cycle from when the users selects their destination and their route is calculated till they complete their travel.\\\\}
				
				\begin{figure}[H]
					\includegraphics[scale=0.5]{Diagrams/Sequence_Diagram_Navigation.JPG}
					\caption{Navigation Sequence Diagram}	
				\end{figure}
				{This Navigation Sequence Diagram shows how Navigation objects operate with one another and in what order. It does this using a construct of messages to and from modules. this diagram shows object interactions arranged in time sequence.\\\\}
				
				\begin{figure}[H]
                     \includegraphics[height=0.8\textheight]{Diagrams/Navigation_State.png}
                     \caption{Route Access State Diagram}
					 \label{fig:navigation_state}
			    \end{figure}
			    {The states of the navigation module concern three main functions. Namely, getRoute, deleteRoute, and saveRoute. Before any one function is called the system will be in a state of awaiting a user request.\\
			    
			    getRoute:\\
			    When a user requests a route the client will send a request to the application server to query a database for the route. The application server will idle and constantly check for a response. If the route was previously saved it is returned to the client. Otherwise the server calculates the route according to user preferences and returns the route to the user.\\
			    
			    deleteRoute:\\
			    When a request for the deletion of a route is made the server enters a confirmation state which awaits user input in order to delete the route.\\
			    
			    saveRoute:\\
			    The system queries the database of saved routes to ensure a route isn't saved twice. If there are no duplicates, the server attempts to save the route to the database until it succeeds.\\
			    }
			    
			    \begin{figure}[H]
                     \includegraphics[width=\textwidth]{Diagrams/Navigation_UseCase.png}
                     \caption{UseCase Diagram}
					 \label{fig:navigation_UseCase}
			    \end{figure}
			    {getRoute, deleteRoute, saveRoute, and modifyRoute extend route access in the navigation module. The users interact with routeAccess to increase "plugability" of available functions. When getRoute is called it either instantiates calculateSimplestRoute or calculateShortestRoute depending on the user preferences.\\\\}
				
				
			
			
			\subsection{Notification Subsystem}\label{subsec:notification}
			
				\begin{figure}[h!]
					\includegraphics[scale=0.5]{Diagrams/Class_Diagram_Notifications.JPG}
					\caption{Notifications Subsystem Class Diagram}	
				\end{figure}
				{The class diagram depicts how the Notification Subsystem implements itself based on the logic of the Factory Method design pattern. Notification Request contains a Notification object. Notification Request is created in Notification Manager (which acts as the factory) where it is stored in a list to be processed. Requests are then checked in the Manager for validity and existence of required users and, after passing validation, are sent to the user.\\\\}
			
				\begin{figure}[h!]
					\includegraphics[scale=0.5]{Diagrams/Activity_Diagram_Notifications.JPG}
					\caption{Notifications Subsystem Activity Diagram}	
				\end{figure}
				{The activity diagram of the Notifications Subsystem depicts how the Notifications Manager is the centre point of system. It is stimulated by an external event which causes it to create a Notification Request, which in turn instantiates a new Notification. The Notification Request then asks the Notification Manager if it can be sent to a user. The Manager then checks if the user is in the system and verifies the request. If it is a valid request, it is sent to the user via SMS or email. If not, it is cancelled and pending requests are processed.\\\\}
				
				\begin{figure}[h!]
					\includegraphics[scale=0.5]{Diagrams/Sequence_Diagram_Notifications.JPG}
					\caption{Notifications Subsystem Sequence Diagram}	
				\end{figure}
				{The above sequence diagram of the Notification Subsystem visualises how the Notifications process begins with the Notification Manager. The process then shifts to the creation of the Notification Request which instantiates a Notification object and returns back to the Manager. Manager then takes time to validate the request and, if granted, dispatches it to User.\\\\}
				
				\begin{figure}[h!]
					\includegraphics[scale=0.5]{Diagrams/State_Diagram_Notifications.JPG}
					\caption{Notifications Subsystem State Diagram}	
				\end{figure}
				{The state diagram of the Notifications subsystem displays how the Notification Manager starts without a Notification request. It then proceeds to create said request which in turn creates a Notification. Once the Notification Request is presented for checking, the system shifts from a creating state to a validation state. If validation succeeds, the system changes to a dispatching state and the Notification is sent to the User.\\\\}
				
%				\begin{figure}[h!]
%					\includegraphics[scale=0.5]{Diagram/Use_Case_Diagram_Notifications.JPG}
%					\caption{Notifications Subsystem Use Case Diagram}	
%				\end{figure}
				{The use case diagram of the Notifications subsystem depicts how important the Notification Manager is for the functionality and goal of the subsystem. It shows how all the important actions take place due to the Manager and how without it the system would collapse. It also shows how the user is hardly involved in the subsystem except to receive validated Notifications.\\\\}
			    
			%\subsection{User Subsystem}\label{subsec:users}
			%\subsection{Notification Subsystem}\label{subsec:notification}

			\newpage
			\subsection{Points of Interest Subsystem}\label{subsec:points of interest}
				\begin{figure}[H]
		 \includegraphics[height=0.8\textheight]{Diagrams/PoI_Class.png}
                     \caption{Point of Interest Class Diagram}
					 \label{fig:PoI_class}
			    \end{figure}

			\begin{figure}[H]
		 \includegraphics[height=0.8\textheight]{Diagrams/PoI_UseCase.png}
                     \caption{Point of Interest Use Case Diagram}
					 \label{fig:PoI_UseCase}
			    \end{figure}

		\newpage

		\section{System assessment}\label{sec:assessment }	
			\subsection{Technology choices}\label{subsec:tech choices}
			
							{Moble application Technology\\
For the mobile app technology we have chosen to react native because 
With React Native, you don't build a “mobile web app”, an “HTML5 app”, or a “hybrid app”.
 You build a real mobile app that's indistinguishable from an app built using Objective-C or Java.
 React Native uses the same fundamental UI building blocks as regular iOS and Android apps.
 You just put those building blocks together using JavaScript and React.
React Native lets you build your app faster. Instead of recompiling, you can reload your app instantly. With hot reloading,
 you can even run new code while retaining your application state.\\

Web Api\\
For the web api we have selected node.js becauese,it is one of the best latest technologies that work best for mobile app development, Node.js uses an event-driven,
 non-blocking I/O model that makes it lightweight and efficient. Node.js' package ecosystem,
 npm, is the largest ecosystem of open source libraries in the world.\\
 
Database Server\\
For this System we have chosen Mongodb because it is the best noSQL database technology out there and it uses JSON-like documents with schemas.
It can be used as a file system with load balancing and data replication features over multiple machines for storing files.
and because if its architecture it is built to handle millions of records without compromising on performance.
.\\\\}
			
			
			\subsection{Quality and Feasibility of Design}\label{subsec:quality}

	 \cleardoublepage
	 \section{Appendices}\label{sec:appendices}
 		
	


\end{document}
